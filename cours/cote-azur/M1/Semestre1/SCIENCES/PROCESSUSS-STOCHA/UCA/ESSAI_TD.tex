\documentclass{article}
\usepackage{amsmath}
\usepackage{amssymb}

\begin{document}

\textbf{Propriété de Martingale :}

La propriété de martingale stipule que, pour tout \(n\), l'espérance conditionnelle de \(u(X_{n+1})\) sachant \(\mathcal{F}^{X}_n\) doit être égale à \(u(X_n)\). Cela s'exprime comme suit :
\[E(u(X_{n+1}) | \mathcal{F}^{X}_n) = u(X_n)\]

Dans le cas présent, \(u(X_n) \in E1\) est un vecteur propre associé à la valeur propre 1, ce qui signifie que \(Pu(X_n) = u(X_n)\). Nous pouvons donc écrire la propriété de martingale comme suit :
\[E(u(X_{n+1}) | \mathcal{F}^{X}_n) = P u(X_n) = u(X_n)\]

$\mathcal{T}$

\textbf{Intégrabilité Uniforme :}

Pour montrer que le processus est uniformément intégrable, nous devons vérifier que \(\sup_{n} E(\|u(X_n)\|) < \infty\). Cela signifie que l'espérance du module du processus est bornée pour tout \(n\).
\[ \sup_{n} E(\|u(X_n)\|) < \infty \]

Puisque \(u(X_n)\) est un vecteur, on considère la norme euclidienne pour mesurer la "taille" du vecteur. La propriété devient donc :
\[ \sup_{n} E(\|P u(X_n)\|) < \infty \]

Ce qui est vrai puisque \(P\) est une matrice de transition et la norme des vecteurs reste bornée lors de l'application de \(P\).

En conclusion, le processus \((u(X_n))_{n \in \mathbb{N}}\) est une martingale uniformément intégrable par rapport à la filtration \(\mathcal{F}^{X}_n\). $\mathbb{R}$


% Exemple simple avec réunion disjointe et indice en bas
\[
\bigcup\limits_{i=1}^n A_i
\]

% Exemple avec réunion disjointe et une fonction plus complexe sous le symbole
\[
\bigcup\limits_{i=1}^n \{ x \in X \mid P_i(x) \}
\]

% Exemple avec réunion disjointe d'ensembles indexés par des naturels
\[
\bigcup\limits_{k \in \mathbb{N}} A_k
\]

% Exemple avec réunion disjointe d'ensembles indexés par un ensemble de conditions
\[
\bigcup\limits_{\substack{i \in I \\ j \in J}} B_{ij}
\]
% Exemple avec une fonction de caractérisation sous la réunion
\[
\bigcup\limits_{i=1}^n \{ x \in X \mid f_i(x) = 0 \}
\]
% Exemple avec une fonction de caractérisation sous la réunion
\[
\bigcup\limits_{i=1}^n \{ x \in X \mid f_i(x) = 0 \}
\]


\end{document}
