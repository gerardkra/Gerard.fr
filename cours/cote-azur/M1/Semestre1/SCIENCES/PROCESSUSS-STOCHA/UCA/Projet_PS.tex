\documentclass[a4paper,12pt]{report}
\usepackage[utf8]{inputenc}
\usepackage[T1]{fontenc}
\usepackage[french]{babel}
\usepackage{listings}  % Pour ajouter la coloration syntaxique
\usepackage{color}
\usepackage{geometry}
\usepackage{graphicx}
\usepackage{tikz}
\usepackage{titlesec}
\usetikzlibrary{shapes,positioning}
\usepackage[most]{tcolorbox}
\renewcommand{\thesection}{\arabic{section}}
% Définir les marges
\geometry{top=2cm, bottom=2cm, left=2cm, right=2cm}

% Commande pour le thème encadré
\newtcbox{\mythemebox}{
  colframe=black,
  colback=white,
  boxrule=2pt,
  arc=6pt,
  boxsep=10pt,
  left=4pt,
  right=4pt
}
% Configuration du style du code
\lstdefinestyle{mystyle}{
    language=Python,
    basicstyle=\ttfamily,
    keywordstyle=\color{blue},
    commentstyle=\color{green},
    numbers=left,
    numberstyle=\tiny\color{gray},
    breaklines=true,
    showstringspaces=false,
    frame=single
}
\begin{document}

% Page de garde
\begin{titlepage}
  \begin{center}
    % Logo de l'université
    \includegraphics[width=5cm]{LogoU.png}

     % Thème encadré
  \vspace{3cm}
  
        
   \begin{LARGE}
    \textbf{SUJET D'ANALYSE}
   \end{LARGE}
   
   \vspace{1cm}
  \fbox{
            \parbox{\textwidth}{
                \centering
                \textbf{Etude du fonctionnement de l’algorithme de classement des pages Web utilisé par le moteur de recherche Google}
            }
        }
  
    % Noms des parties prenantes
    \vspace{5cm}
    Présenté par:\\
    \textbf{Kouamé Gérard Kra}\\
    \textbf{Seyed Hossein Seyedi Nahrmiani}\\
    \textbf{Koffi Roland Wotobe}\\

    \vspace{1cm}
   Sous la direction de:\\
    \textbf{ M Rubenthaler}

    \vfill
    % Date
    \today
  \end{center}
\end{titlepage}

    % PageRanks algorithm
   
\begin{center}


\begin{figure}[ht]
\begin{center}
\includegraphics[scale=0.25]{PageRanks.png}
\caption{Une simple illustration de l'algorithme Pagerank. Le pourcentage indique l'importance perçue, et les flèches représentent les hyperliens.}{Source:https://en.wikipedia.org/wiki/PageRank}
\end{center}
\end{figure}

\end{center}
% Table des matières
\tableofcontents

\section{INTRODUCTION}
Dans le vaste espace numérique qu'est l'Internet, la quête d’informations commence souvent par une simple recherche. Etant donné qu'il y a une multitude d'informations sur le web, le besoin de gagner du temps et la pertinence des informations deviennent alors nécessaires pour chaque requête. C'est l'une des préoccupations fondamentales qui incita Larry Page et Sergueï Brin à concevoir le moteur de recherche Google. Depuis sa conception en 1998, Google a connu plusieurs améliorations et cela fait de lui aujourd'hui le moteur de recherche le plus utilisé au plan mondial.
Cependant, la plupart des réformes de Google demeurent jusqu'à présent des secrets. Par contre, selon l'un des articles publiés par les fondateurs [1], le pilier de son succès est une judicieuse modélisation mathématique que nous retraçons ici.
 
\section{DEFINITIONS DES MOTS-CLES}
    Dans cette section nous définissons quelques mots essentiels autour desquels notre présentation va s'articuler:
    \begin{itemize}
        \item Classement des pages Web : Le processus d'ordonnancement des résultats de recherche en fonction de leur pertinence et de leur importance.
        \item Importance d'une page web : La mesure de la pertinence et de l'autorité d'une page Web dans le contexte d'une recherche en ligne.
        \item Graphe du Web : Représentation graphique des pages Web interconnectées, où les différents noeuds représentent les pages web et les flèches entre les pages représentent les liens entre ces dernières.
        \item Algorithmes de recherche : De façon générale un algorithme est une suite finie et non ambiguë d'instructions et d’opérations permettant de résoudre un problème concret. En particulier un algorithme de recherche est une méthode informatique utilisée pour trouver des informations précises dans un ensemble de données.
    \end{itemize}
    Dans la suite de cette présentation, nous utiliserons le graphe du web ci-dessous.
\begin{figure}[ht]
\begin{center}
\includegraphics[scale=0.5]{file1bis.png}
\caption{figure}{Exemple de graphe du web}
\label{fig1}
\end{center}
\end{figure}

 
    \section{ORIGINE DE PAGERANK}
Le modèle PageRank est un modèle beaucoup plus raffiné utilisé par Google pour la classification des pages web.
Google fournit des résultats de recherche qui correspondent aux termes de recherche de l'utilisateur. En coulisse, Google maintient un classement parmi les sites web pour s'assurer que les sites "meilleurs" ou "plus importants" apparaissent tôt dans les résultats de recherche. Au lieu de résoudre le problème dans son ensemble en une seule fois, ce classement est d'abord établi de manière globale (indépendamment des termes de recherche), et seulement plus tard, les sites web correspondant à la requête de recherche sont triés selon un certain classement. Dans cette section, nous nous concentrons sur la partie classement, nous notons les pages web par $P_1,P_2,P_3,\ldots,P_{n}$ et nous écrivons j$\rightarrow$i si la page $P_{j}$ cite la page $P_{i}$.
 
\subsection{Approche récursive}
 

L'approche récursive consiste à dire qu'une page $P_i$ paraît importante si beaucoup de pages importantes la citent. Ceci nous mène à définir l'importance $m_i$ de manière récursive comme suit :
\begin{equation}
    \ m_i= \sum_{j\rightarrow i}\frac{1}{l_j}m_j
    \label{eq:1}
\end{equation}
où $l_j$ dénote le nombre de liens émis.\\
Remarque: 
\begin{enumerate}
\item[•] Ici le poids du vote j$\rightarrow$i est proportionnel au poids $m_j$ de la page émettrice.
\item[•] (1) est un système de n équations linéaires à n inconnues. Dans notre exemple, où n=10, il est déjà pénible à résoudre à la main, mais encore facile sur ordinateur. Pour les graphes beaucoup plus grands nous aurons besoin de méthodes spécialisées.
\end{enumerate}
\subsection{Promenade aléatoire} 
Avant de tenter de résoudre l'équation (1), essayons d'en développer une intuition. Pour ceci imaginons un surfeur aléatoire qui se balade sur internet en cliquant sur les liens au hasard. Comment évolue sa position?
À titre d'exemple, supposons que notre surfeur démarre au temps $t=0$ sur la page $P_1$. De là les liens pointent vers $P_2$, $P_3$, $P_4$, $P_5$ donc au temps $t=1$ le surfeur se retrouvent sur chacune de ces pages avec probabilité $\frac{1}{4}$. Voici les probabilités suivantes (arrondies à $10^{-3}$ près) :

Voici le code:

% Insérer le code Python
\begin{verbatim}
import numpy as np

# le nombre de page
n = 12

# Poids ou importance des pages
m = np.array([2, 1, 1, 1, 3, 1, 2, 1, 2, 1, 1, 1])

# Le nombre de liens sortants de chaque page
L = np.array([1, 1, 3, 3, 4, 10, 3, 2, 3, 1, 2, 1])

# matrice de transition
P = np.zeros((n, n))

for i in range(n):
    for j in range(n):
        if L[j] != 0:
            P[i, j] = m[j] / L[j]

# la loi initiale
initial_distribution = np.ones(n) / n

# Algorithme de puissances matricielles pour résoudre l'équation m = m * P
iterations = 30
current_distribution = initial_distribution.copy()
for _ in range(iterations):
    current_distribution = np.dot(current_distribution, P)

# Afficher la distribution stable (distribution finale)
print("la loi stationaire:")
print(current_distribution)
\end{verbatim}
        \begin{center}
        \begin{tabular}{c c c c c c c c c c c} 
& $P_1$ & $P_2$ & $P_3$ & $P_4$ & $P_5$ & $P_6$ & $P_7$ & $P_8$ & $P_9$ & $P_{10}$\\
        $t = 0$ & $1.00$ & $.000$ & $.000$ & $.000$ & $.000$ & $.000$ & $.000$ & $.000$ & $.000$ & $.000$ \\
        $t = 1$ & $.000$ & $.25$ & $.25$ & $.25$ & $.25$ & $.000$ & $.000$ & $.000$ & $.000$ & $.000$ \\
         $t = 2$ & $.375$ & $.125$ & $.125$ & $.125$ & $.000$ & $.083$ & $.083$ & $.083$ & $.000$ & $.000$ \\
         $t = 3$ & $.229$ & $.156$ & $.156$ & $.156$ & $.177$ & $.000$ & $.083$ & $.000$ & $.042$ & $.000$ \\
         $t = 4$ & $.234$ & $.135$ & $.135$ & $.146$ & $.151$ & $.059$ & $.059$ & $.059$ & $.01$ & $.01$ \\
 
         $t = 5$ & $.238$ & $.132$ & $.126$ & $.129$ & $.120$ & $.05$ & $.115$ & $.05$ & $.037$ & $.003$ \\
 
$\vdots$  \\
 
         $t = 25$ & $.182$ & $.098$ & $.095$ & $.106$ & $.193$ & $.064$ & $.135$ & $.064$ & $.051$ & $.013$ \\
 
         $t = 26$ & $.181$ & $.098$ & $.095$ & $.106$ & $.193$ & $.064$ & $.135$ & $.064$ & $.051$ & $.013$
        \end{tabular}
        \end{center}
    Remarquons que selon la position initiale du surfeur, le nombre d'itérations peut varier mais tout en gardant la distribution stationnaire.
    C'est le cas de l'exemple ci-dessous où le surfeur commence initialement sur la page $P_5$. De là les liens pointent vers $P_6$, $P_7$, $P_8$ donc au temps $t=1$ le surfeur se retrouvent sur chacune de ces pages avec probabilité $\frac{1}{3}$. Voici les probabilités suivantes (arrondies à $10^{-3}$ près) :
        \begin{center}
        \begin{tabular}{c c c c c c c c c c c} 
& $P_1$ & $P_2$ & $P_3$ & $P_4$ & $P_5$ & $P_6$ & $P_7$ & $P_8$ & $P_9$ & $P_{10}$\\
        $t = 0$ & $.000$ & $.000$ & $.000$ & $.000$ & $.000$ & $.000$ & $.000$ & $.000$ & $.000$ & $.000$ \\
        $t = 1$ & $.000$ & $.25$ & $.25$ & $.25$ & $.25$ & $.000$ & $.000$ & $.000$ & $.000$ & $.000$ \\
         $t = 2$ & $.375$ & $.125$ & $.125$ & $.125$ & $.000$ & $.083$ & $.083$ & $.083$ & $.000$ & $.000$ \\
         $t = 3$ & $.229$ & $.156$ & $.156$ & $.156$ & $.177$ & $.000$ & $.083$ & $.000$ & $.042$ & $.000$ \\
         $t = 4$ & $.234$ & $.135$ & $.135$ & $.146$ & $.151$ & $.059$ & $.059$ & $.059$ & $.01$ & $.01$ \\
 
         $t = 5$ & $.238$ & $.132$ & $.126$ & $.129$ & $.120$ & $.05$ & $.115$ & $.05$ & $.037$ & $.003$ \\
 
$\vdots$\\
 
         $t = 25$ & $.182$ & $.098$ & $.095$ & $.106$ & $.193$ & $.064$ & $.135$ & $.064$ & $.051$ & $.013$ \\
 
         $t = 26$ & $.181$ & $.098$ & $.095$ & $.106$ & $.193$ & $.064$ & $.135$ & $.064$ & $.051$ & $.013$
        \end{tabular}
        \end{center}

Le modèle du surfeur aléatoire peut sembler étonnant, mais en absence d'information plus précise, le recours aux considérations probabilistes se révèle souvent très utile !
\subsection{Loi de transition} 
Comment formaliser la diffusion illustrée ci-dessus ? Supposons qu'au temps $t$ notre surfeur aléatoire se trouve sur la page $P_j$ avec une probabilité $p_j$. La probabilité de partir de $P_j$ et de suivre le lien j$\rightarrow$i est alors $\frac{1}{l_j}m_j$. La probabilité d'arriver au temps $t+1$ sur la page $P_i$ est donc
\begin{equation}
    \ p'_i= \sum_{j\rightarrow i}\frac{1}{l_j}m_j
    \label{eq:2}
\end{equation}
Étant donnée la distribution initiale $p$, la loi de transition $(2)$ définit la distribution suivante $p'$ telle que $p' = T(p)$. C'est ainsi que l'on obtient la ligne $t+1$ à partir de la ligne $t$ dans nos exemples. (En théorie des probabilités, ceci est appelé une chaîne de Markov.) La mesure stationnaire est caractérisée par l'équation d'équilibre $m = T(m)$, qui correspond justement à notre équation de départ $(1)$.


 
 
\section{PageRank}
Le PageRank ou PR est un algorithme d'analyse de liens concourant au système de classement des pages Web utilisé par le moteur de recherche Google. Il mesure quantitativement la popularité d'une page web. Le PageRank n'est qu'un indicateur parmi d'autres dans l'algorithme qui permet de classer les pages du Web dans les résultats de recherche de Google.
\subsection{Fonctionnement de PageRank}
L'idée principale derrière PageRank est de mesurer l'importance ou la popularité d'une page en analysant les liens entrants et sortants de cette dernière.
Pour ce faire, PageRank se base sur quelques concepts que nous définissons ci-dessous.
 
\begin{itemize}
\item \textbf{La notion de vote} : PageRank considère un lien d'une page I vers une page J comme un vote de la page I pour la page J. Plus il y a de liens pointant vers une page, plus cette page est considérée comme importante ou influente.
 
\item \textbf{La pondération des votes} : Tous les votes ne sont pas égaux. L'importance d'une page est déterminée par le nombre et la qualité des pages qui pointent vers elle. Si une page importante (avec un score élevé elle-même) pointe vers une autre page, ce vote sera plus significatif qu'un lien provenant d'une page moins importante.
 
\item \textbf{L'algorithme itératif} : PageRank fonctionne de manière itérative. Au départ, chaque page se voit attribuer un score égal. Ensuite, ces scores sont mis à jour itérativement en fonction des votes des pages qui y pointent et de leur propre importance. Ce processus se répète jusqu'à ce que les scores convergent vers une valeur stable (cette valeur stable est en faite la distribution stationnaire de la chaîne de Markov associée à notre graphe du web ci-dessus).
 
\item \textbf{La matrice de transition} : On peut représenter les liens entre les pages sous forme d'une matrice de transition, où chaque élément de la matrice représente la probabilité de passer d'une page à une autre en suivant les liens. En appliquant des calculs matriciels itératifs, on peut calculer les scores de PageRank pour chaque page.
 
\item \textbf{Damping factor et surfer aléatoire} : PageRank utilise également un facteur d'amortissement pour modéliser le comportement des utilisateurs qui pourraient simplement naviguer de manière aléatoire sans suivre de liens. Cela permet d'éviter les boucles infinies et de garantir la convergence de l'algorithme. (Dans notre graphe du web on peut avoir des sites qui ne pointent vers aucun autre site. Le damping factor permet donc de relier tout notre notre graphe malgré l'existence de tels sites web ainsi on est toujours sûr de pouvoir se déplacer sur le graphe)
 
\end{itemize}
 
\section{MISE EN ŒUVRE}
Dans cette partie nous allons mettre en oeuvre l'algorithme sur le graphe qui nous sert d'exemple. Nous effectuerons donc le calcul des scores PageRank, une visualisation graphique des scores PageRank et nous parlerons aussi de la convergence de l'algorithme.
 
\end{document}
