\documentclass{report}
\usepackage{verbatim}  % Pour l'environnement verbatim
\usepackage{lipsum}

\begin{document}

\tableofcontents

% Ajouter le tableau "Programme 1" avec le code Python
\begin{table}[h]
  \centering
  \begin{tabular}{|p{\textwidth}|}
    \hline
    \textbf{Programme 1:} \\
    \begin{verbatim}
import numpy as np

# le nombre de page
n = 12

# Poids ou importance des pages
m = np.array([2, 1, 1, 1, 3, 1, 2, 1, 2, 1, 1, 1])

# Le nombre de liens sortants de chaque page
L = np.array([1, 1, 3, 3, 4, 10, 3, 2, 3, 1, 2, 1])

# matrice de transition
P = np.zeros((n, n))

for i in range(n):
    for j in range(n):
        if L[j] != 0:
            P[i, j] = m[j] / L[j]

# la loi initiale
initial_distribution = np.ones(n) / n

# Algorithme de puissances matricielles pour résoudre l'équation m = m * P
iterations = 30
current_distribution = initial_distribution.copy()
for _ in range(iterations):
    current_distribution = np.dot(current_distribution, P)

# Afficher la distribution stable (distribution finale)
print("la loi stationaire:")
print(current_distribution)
    \end{verbatim} \\
  \end{tabular}
\end{table}

% Le reste du document
\chapter{Introduction}

\end{document}
