\documentclass{article}
\usepackage{amsmath}
\usepackage{amssymb}

\begin{document}

Soit $P(n)$ la proposition que l'équation
\[a^n - b^n = (a - b) \sum_{k=0}^{n-1} a^k b^{n-1-k}\]
est vraie pour tout $n \in \mathbb{N}^*$.

\textbf{Base de l'induction (n = 1):}

Vérifions d'abord si $P(1)$ est vraie :
\[a^1 - b^1 = a - b\]
L'équation est vérifiée lorsque $n = 1$.

\textbf{Hypothèse d'induction:}

Supposons maintenant que $P(k)$ est vraie pour un certain $k \in \mathbb{N}^*$, c'est-à-dire que
\[a^k - b^k = (a - b) \sum_{i=0}^{k-1} a^i b^{k-1-i}\].

\textbf{Étape de l'induction:}

Nous devons montrer que $P(k+1)$ est également vraie. Considérons l'expression $a^{k+1} - b^{k+1}$ :
\[a^{k+1} - b^{k+1} = a \cdot a^k - b \cdot b^k\]

Factorisons $a$ dans le premier terme et $b$ dans le deuxième terme :
\[= a \cdot a^k - b \cdot b^k = a^k(a - b) - b^k(a - b)\]

Factorisons $a - b$ :
\[= (a - b) \cdot a^k - b^k(a - b)\]

Regroupons les termes avec $(a - b)$ :
\[= (a - b) \cdot (a^k - b^k)\]

Maintenant, nous pouvons utiliser l'hypothèse d'induction :
\[= (a - b) \cdot [(a - b) \sum_{i=0}^{k-1} a^i b^{k-1-i}]\]

Distribuons la somme et simplifions :
\[= (a - b) \sum_{i=0}^{k-1} a^{i+1} b^{k-1-i} - (a - b) \sum_{i=0}^{k-1} a^i b^{k-i}\]

Regroupons les termes de la somme :
\[= (a - b) \sum_{i=1}^{k} a^i b^{k-i} - (a - b) \sum_{i=0}^{k-1} a^i b^{k-i}\]

Les termes se simplifient :
\[= (a - b) [a^k b^0 - a^0 b^k]\]
\[= (a - b) [a^k - b^k]\]

En utilisant l'hypothèse d'induction, nous obtenons :
\[= (a - b) \sum_{i=0}^{k} a^{i} b^{k-i}\]

Cela montre que $P(k+1)$ est également vraie.

Par conséquent, par le principe de l'induction, l'équation est démontrée pour tout $n \in \mathbb{N}^*$.

\end{document}
