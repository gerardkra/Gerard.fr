\documentclass{article}
\usepackage[utf8]{inputenc}
\usepackage[french]{babel}
\usepackage[T1]{fontenc}
\usepackage{pgfplots}
\usepackage{amsmath}
\usepackage{amssymb}
\author{ROMEO}
\title{Evaluation n°1}
\begin{document}
\maketitle
\emph{\underline{Exercice 1}}\\
Déterminer les ensembles de définition des fonctions suivantes:\\
\begin{enumerate}
\item[a.] $f(x)=x^3-3x^2+2x-5$
\item[b.] $g(x)=\frac{2x+3}{4x-1}$
\item[c.] $h(x)=4x-1+\sqrt{x}$
\item[d.] $j(x)=\frac{\sqrt{x}}{x}$
\item[e.] $k(x)=\sqrt{x^2-5x+6}$
\end{enumerate}

\emph{\underline{Exercice 2}}\\
\begin{enumerate}
\item Soit $f$ la fonction définie sur $\mathbb{R}$ par $f(x) = 2x +3$.\\

Déterminer l’image de $2$ et le ou les antécédents (s’ils existent) de $4$.
\item Soit $g$ la fonction définie sur $\mathbb{R}$ par $g(x) = x^2-5x$.\\
Déterminer l’image de $-1$ et le ou les antécédents (s’ils existent) de $-6$ et de $-7$.

\end{enumerate}

\emph{\underline{Exercice 3}}\\
soit le graphe d'une fonction $f$ (ligne rouge du graphe) ci-dessous:
\begin{figure}[h]
  \centering
  \begin{tikzpicture}
    \begin{axis}[
      xlabel={$x$},
      ylabel={$y$},
      axis lines=middle,
      grid=both,
      xmax=5,
      xmin=-5,
      ymax=1.5,
      ymin=-1.5,
      xtick={-5,-4,-3,-2,-1,0,1,2,3,4,5},
      ytick={-5,-4,-3,-2,-1,0,1,2,3,4,5},
      legend pos=north east,
    ]
      
      \addplot[blue, domain=-3:6, samples=100] {cos(deg(x))};
    \end{axis}
  \end{tikzpicture}
  \caption{Graphique de la fonction $f$}
\end{figure}
\begin{enumerate}
\item Déterminer l'ensemble de définition de $f$ à travers le graphe.
\item Donner l'image de $-0.5$ et de $0.5$ par la fonction $f$.
\item Donner le ou les antécédents (s’ils existent) de $-6$, de $-0.5$ et de $0.5$ .
\end{enumerate}

\end{document}
