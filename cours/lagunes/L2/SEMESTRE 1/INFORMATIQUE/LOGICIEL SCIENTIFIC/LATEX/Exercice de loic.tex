\documentclass[12pt]{article}
\usepackage[utf8]{inputenc}
\usepackage[french]{babel}
\usepackage[T1]{fontenc}
\usepackage{amsmath}
\usepackage{amssymb}
\usepackage{lmodern}
\usepackage{pstricks}

\author{LOIC TEBAO}
\title{DEVOIR}
\begin{document}
\maketitle
\emph{Exercice} 3.\\
le plan est muni d'une base $(i,j)$. Soit $\overrightarrow{u}$ et $\overrightarrow{v}$ deux vecteurs tels que:\\ $\overrightarrow{u}=4\overrightarrow{i}+\overrightarrow{j}$ et $\overrightarrow{v}=2\overrightarrow{i}-3\overrightarrow{j}$
\begin{enumerate}
\item Donne les coordonnées des vecteurs  $\overrightarrow{u}$ et $\overrightarrow{v}$ dans la base 
$(i,j)$.
\item
\begin{description}
\item{a}- Calculer les déterminants des vecteurs $\overrightarrow{u}$ et $\overrightarrow{v}$ dans la base 
$(i,j)$.
\item{b}- Déduis que les vecteurs $\overrightarrow{u}$ et $\overrightarrow{v}$ forment une base.
\end{description}
\item Soient $\overrightarrow{u}(a,1)$ et $\overrightarrow{v}(1,a)$ deux vecteurs du plan et $a\in \mathbb{R}$.
\begin{description}
\item{a}) Calculer $det(\overrightarrow{u},\overrightarrow{v})$ en fonction de $a$.
\item{b}) Déduis les valeurs de $a$ pour lesquelles les vecteurs $\overrightarrow{u}$ et $\overrightarrow{v}$ sont colinéaires.
\end{description}
\end{enumerate}

\emph{Exercice} 4.\\
On considère la fonction $f$ dont la courbe représentative est ci-contre.
\begin{enumerate}
\item Donne l'ensemble de définition de $f$.
\item Déterminer graphiquement l'image de -3; -1 et 1.
\item Déterminer graphiquement les antécédents de 2.
\item Donner le sens de variation de $f$ et dresser son tableau de variation.\\ Déduire le maximum et le minimum de la fonction $f$.
\end{enumerate}

\end{document}
