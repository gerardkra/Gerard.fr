\documentclass{article}
\usepackage{amsmath}

\begin{document}

\textbf{2):} Soit la fonction \(f(x) = \sqrt{x+2} - \sqrt{x}\) définie sur \([0,+\infty[\). Montrons que pour tout \(x\) dans \([0,+\infty[\), \(0 \leq f(x) \leq \frac{2}{\sqrt{x}}\).\\



\textbf{Étape 1 :} Montrons que \(0 \leq f(x)\)

La fonction \(f(x) = \sqrt{x+2} - \sqrt{x}\) est définie sur l'intervalle \([0,+\infty[\), donc \(x\) est toujours positif ou nul. Dans ce cas, \(\sqrt{x+2}\) et \(\sqrt{x}\) sont toujours positifs, car la racine carrée d'un nombre positif est positive. Ainsi, \(f(x) = \sqrt{x+2} - \sqrt{x} \geq 0\) car \(\sqrt{x+2} \geq \sqrt{x}\) (car \(x+2 \geq x\) puisque \(x\) est positif).\\


\textbf{Étape 2 :} Montrons que \(f(x) \leq \frac{2}{\sqrt{x}}\)

Remarquons que \(f(x) = \sqrt{x+2} - \sqrt{x}\) peut être réarrangée comme suit :

\[f(x) = \sqrt{x+2} - \sqrt{x} = \frac{(\sqrt{x+2} - \sqrt{x})(\sqrt{x+2} + \sqrt{x})}{\sqrt{x+2} + \sqrt{x}} = \frac{(x+2) - x}{\sqrt{x+2} + \sqrt{x}} = \frac{2}{\sqrt{x+2} + \sqrt{x}}\]

Puisque \(x\) est positif, on a \(\sqrt{x} > 0\) et \(\sqrt{x+2} > 0\), donc \(\sqrt{x} + \sqrt{x+2} > \sqrt{x} > 0\). Ainsi, \(\frac{1}{\sqrt{x} + \sqrt{x+2}} < 1\).

En utilisant cette information, nous pouvons écrire :

\[f(x) = \frac{2}{\sqrt{x+2} + \sqrt{x}} < \frac{2}{\sqrt{x}}\]

Donc, \(f(x) \leq \frac{2}{\sqrt{x}}\).\\


Par conséquent, \(\forall x \in, \([0,+\infty[\), \(0 \leq f(x) \leq \frac{2}{\sqrt{x}}\).

\end{document}
