\documentclass[9pt,a4paper]{article}
\usepackage[utf8]{inputenc}
\usepackage[french]{babel}
\usepackage{fancyhdr}%en tête et pied de page
\pagestyle{fancy}%en-tête et pied de page
\usepackage[utf8]{inputenc}%gestion des accents
\usepackage{amsmath}
\usepackage{amssymb}
\def\nbR{\ensuremath{\mathrm{I\! R}}}
\usepackage{fancybox}

\renewcommand{\headrulewidth}{1pt}
\fancyhead[L]{Université des Lagunes\\Licence 2 : Mathématiques}%en-tête de page côté gauche
\fancyhead[R]{République de Côte d'Ivoire\\Année universitaire : 2021/2022}%en-tête de page côté droit

\renewcommand{\footrulewidth}{1pt}%%en-tête et pied de page

\begin{document}
{\fontfamily{phv}\selectfont%
\baselineskip=2pt$$CONTR\hat{O}LE \ CONTINU :$$\\$$INT\acute{E}GRALES\ G\acute{E}N\acute{E}RALIS\acute{E}ES$$\\$$Dur\acute{e}e : 1\ heure$$
\begin{center}%tracer une droite horizontale début
\rule{0.999\linewidth}{1pt} %taille
\end{center}%fin
{\scriptsize La rigueur des raisonnenments ainsi que la clarté et la qualité de la rédaction seront prises en compte dans l'évaluation.}\\

\medskip\shadowbox{Exercice 1}\\

\scriptsize En utilisant la définition de la convergence d'une intégrale généralisée , montrer la convergence et donner la valeur de l'intégrale :\\
\normalsize$$\int_{0}^{1}\frac{1}{\sqrt{1 -x}}dx$$\\

\medskip\shadowbox{Exercice 2}\\
\scriptsize Montrer que l'intégrale généralisée\\\normalsize$$\int_{0}^{+\infty}\frac{\sqrt{x}\sin(\frac{1}{x^2})}{\ln(1 +x)}dx$$\\
\scriptsize est convergente.\\


\medskip\shadowbox{\normalsize Exercice 3}\\
\scriptsize Montrer que l'intégrale généralisée\\\normalsize$$\int_{0}^{+\infty}{\biggl(1 + \ln\biggl(\frac{x}{x + 1}\biggr)\biggr)}dx$$ \\
\scriptsize est divergente.


\medskip\shadowbox{\normalsize Exercice 4}\\
\scriptsize Soit \small$\alpha$$ \ \in$ \ $\mathbb{R^*_+}$ (\small$\alpha$ $>0$) Étudier selon la valeur de \small$\alpha$ la convergence de l'intégrale généralisée\\
$$\int_{0}^{+\infty}\frac{x - \sin(x)}{x^\alpha}dx$$\\
On rappelle qu'au voisinage de 0 , $\sin(x)$ = ${x}$ - $\frac{x^3}{6}+ o(x^3). $

\end{document}